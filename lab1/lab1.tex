\documentclass[11pt]{article}
    \usepackage[utf8]{inputenc}
    \usepackage[T1,T2A]{fontenc}
    \usepackage[russian]{babel}


    \usepackage[breakable]{tcolorbox}
    \usepackage{parskip} % Stop auto-indenting (to mimic markdown behaviour)
    

    % Basic figure setup, for now with no caption control since it's done
    % automatically by Pandoc (which extracts ![](path) syntax from Markdown).
    \usepackage{graphicx}
    % Maintain compatibility with old templates. Remove in nbconvert 6.0
    \let\Oldincludegraphics\includegraphics
    % Ensure that by default, figures have no caption (until we provide a
    % proper Figure object with a Caption API and a way to capture that
    % in the conversion process - todo).
    \usepackage{caption}
    \DeclareCaptionFormat{nocaption}{}
    \captionsetup{format=nocaption,aboveskip=0pt,belowskip=0pt}

    \usepackage{float}
    \floatplacement{figure}{H} % forces figures to be placed at the correct location
    \usepackage{xcolor} % Allow colors to be defined
    \usepackage{enumerate} % Needed for markdown enumerations to work
    \usepackage{geometry} % Used to adjust the document margins
    \usepackage{amsmath} % Equations
    \usepackage{amssymb} % Equations
    \usepackage{textcomp} % defines textquotesingle
    % Hack from http://tex.stackexchange.com/a/47451/13684:
    \AtBeginDocument{%
        \def\PYZsq{\textquotesingle}% Upright quotes in Pygmentized code
    }
    \usepackage{upquote} % Upright quotes for verbatim code
    \usepackage{eurosym} % defines \euro

    \usepackage{iftex}
    \ifPDFTeX
        \usepackage[T1]{fontenc}
        \IfFileExists{alphabeta.sty}{
              \usepackage{alphabeta}
          }{
              \usepackage[mathletters]{ucs}
              \usepackage[utf8x]{inputenc}
          }
    \else
        \usepackage{fontspec}
        \usepackage{unicode-math}
    \fi

    \usepackage{fancyvrb} % verbatim replacement that allows latex
    \usepackage{grffile} % extends the file name processing of package graphics
                         % to support a larger range
    \makeatletter % fix for old versions of grffile with XeLaTeX
    \@ifpackagelater{grffile}{2019/11/01}
    {
      % Do nothing on new versions
    }
    {
      \def\Gread@@xetex#1{%
        \IfFileExists{"\Gin@base".bb}%
        {\Gread@eps{\Gin@base.bb}}%
        {\Gread@@xetex@aux#1}%
      }
    }
    \makeatother
    \usepackage[Export]{adjustbox} % Used to constrain images to a maximum size
    \adjustboxset{max size={0.9\linewidth}{0.9\paperheight}}

    % The hyperref package gives us a pdf with properly built
    % internal navigation ('pdf bookmarks' for the table of contents,
    % internal cross-reference links, web links for URLs, etc.)
    \usepackage{hyperref}
    % The default LaTeX title has an obnoxious amount of whitespace. By default,
    % titling removes some of it. It also provides customization options.
    \usepackage{titling}
    \usepackage{longtable} % longtable support required by pandoc >1.10
    \usepackage{booktabs}  % table support for pandoc > 1.12.2
    \usepackage{array}     % table support for pandoc >= 2.11.3
    \usepackage{calc}      % table minipage width calculation for pandoc >= 2.11.1
    \usepackage[inline]{enumitem} % IRkernel/repr support (it uses the enumerate* environment)
    \usepackage[normalem]{ulem} % ulem is needed to support strikethroughs (\sout)
                                % normalem makes italics be italics, not underlines
    \usepackage{mathrsfs}
    

    
    % Colors for the hyperref package
    \definecolor{urlcolor}{rgb}{0,.145,.698}
    \definecolor{linkcolor}{rgb}{.71,0.21,0.01}
    \definecolor{citecolor}{rgb}{.12,.54,.11}

    % ANSI colors
    \definecolor{ansi-black}{HTML}{3E424D}
    \definecolor{ansi-black-intense}{HTML}{282C36}
    \definecolor{ansi-red}{HTML}{E75C58}
    \definecolor{ansi-red-intense}{HTML}{B22B31}
    \definecolor{ansi-green}{HTML}{00A250}
    \definecolor{ansi-green-intense}{HTML}{007427}
    \definecolor{ansi-yellow}{HTML}{DDB62B}
    \definecolor{ansi-yellow-intense}{HTML}{B27D12}
    \definecolor{ansi-blue}{HTML}{208FFB}
    \definecolor{ansi-blue-intense}{HTML}{0065CA}
    \definecolor{ansi-magenta}{HTML}{D160C4}
    \definecolor{ansi-magenta-intense}{HTML}{A03196}
    \definecolor{ansi-cyan}{HTML}{60C6C8}
    \definecolor{ansi-cyan-intense}{HTML}{258F8F}
    \definecolor{ansi-white}{HTML}{C5C1B4}
    \definecolor{ansi-white-intense}{HTML}{A1A6B2}
    \definecolor{ansi-default-inverse-fg}{HTML}{FFFFFF}
    \definecolor{ansi-default-inverse-bg}{HTML}{000000}

    % common color for the border for error outputs.
    \definecolor{outerrorbackground}{HTML}{FFDFDF}

    % commands and environments needed by pandoc snippets
    % extracted from the output of `pandoc -s`
    \providecommand{\tightlist}{%
      \setlength{\itemsep}{0pt}\setlength{\parskip}{0pt}}
    \DefineVerbatimEnvironment{Highlighting}{Verbatim}{commandchars=\\\{\}}
    % Add ',fontsize=\small' for more characters per line
    \newenvironment{Shaded}{}{}
    \newcommand{\KeywordTok}[1]{\textcolor[rgb]{0.00,0.44,0.13}{\textbf{{#1}}}}
    \newcommand{\DataTypeTok}[1]{\textcolor[rgb]{0.56,0.13,0.00}{{#1}}}
    \newcommand{\DecValTok}[1]{\textcolor[rgb]{0.25,0.63,0.44}{{#1}}}
    \newcommand{\BaseNTok}[1]{\textcolor[rgb]{0.25,0.63,0.44}{{#1}}}
    \newcommand{\FloatTok}[1]{\textcolor[rgb]{0.25,0.63,0.44}{{#1}}}
    \newcommand{\CharTok}[1]{\textcolor[rgb]{0.25,0.44,0.63}{{#1}}}
    \newcommand{\StringTok}[1]{\textcolor[rgb]{0.25,0.44,0.63}{{#1}}}
    \newcommand{\CommentTok}[1]{\textcolor[rgb]{0.38,0.63,0.69}{\textit{{#1}}}}
    \newcommand{\OtherTok}[1]{\textcolor[rgb]{0.00,0.44,0.13}{{#1}}}
    \newcommand{\AlertTok}[1]{\textcolor[rgb]{1.00,0.00,0.00}{\textbf{{#1}}}}
    \newcommand{\FunctionTok}[1]{\textcolor[rgb]{0.02,0.16,0.49}{{#1}}}
    \newcommand{\RegionMarkerTok}[1]{{#1}}
    \newcommand{\ErrorTok}[1]{\textcolor[rgb]{1.00,0.00,0.00}{\textbf{{#1}}}}
    \newcommand{\NormalTok}[1]{{#1}}

    % Additional commands for more recent versions of Pandoc
    \newcommand{\ConstantTok}[1]{\textcolor[rgb]{0.53,0.00,0.00}{{#1}}}
    \newcommand{\SpecialCharTok}[1]{\textcolor[rgb]{0.25,0.44,0.63}{{#1}}}
    \newcommand{\VerbatimStringTok}[1]{\textcolor[rgb]{0.25,0.44,0.63}{{#1}}}
    \newcommand{\SpecialStringTok}[1]{\textcolor[rgb]{0.73,0.40,0.53}{{#1}}}
    \newcommand{\ImportTok}[1]{{#1}}
    \newcommand{\DocumentationTok}[1]{\textcolor[rgb]{0.73,0.13,0.13}{\textit{{#1}}}}
    \newcommand{\AnnotationTok}[1]{\textcolor[rgb]{0.38,0.63,0.69}{\textbf{\textit{{#1}}}}}
    \newcommand{\CommentVarTok}[1]{\textcolor[rgb]{0.38,0.63,0.69}{\textbf{\textit{{#1}}}}}
    \newcommand{\VariableTok}[1]{\textcolor[rgb]{0.10,0.09,0.49}{{#1}}}
    \newcommand{\ControlFlowTok}[1]{\textcolor[rgb]{0.00,0.44,0.13}{\textbf{{#1}}}}
    \newcommand{\OperatorTok}[1]{\textcolor[rgb]{0.40,0.40,0.40}{{#1}}}
    \newcommand{\BuiltInTok}[1]{{#1}}
    \newcommand{\ExtensionTok}[1]{{#1}}
    \newcommand{\PreprocessorTok}[1]{\textcolor[rgb]{0.74,0.48,0.00}{{#1}}}
    \newcommand{\AttributeTok}[1]{\textcolor[rgb]{0.49,0.56,0.16}{{#1}}}
    \newcommand{\InformationTok}[1]{\textcolor[rgb]{0.38,0.63,0.69}{\textbf{\textit{{#1}}}}}
    \newcommand{\WarningTok}[1]{\textcolor[rgb]{0.38,0.63,0.69}{\textbf{\textit{{#1}}}}}


    % Define a nice break command that doesn't care if a line doesn't already
    % exist.
    \def\br{\hspace*{\fill} \\* }
    % Math Jax compatibility definitions
    \def\gt{>}
    \def\lt{<}
    \let\Oldtex\TeX
    \let\Oldlatex\LaTeX
    \renewcommand{\TeX}{\textrm{\Oldtex}}
    \renewcommand{\LaTeX}{\textrm{\Oldlatex}}
    % Document parameters
    % Document title
    \title{lab1}
    
    
    
    
    
% Pygments definitions
\makeatletter
\def\PY@reset{\let\PY@it=\relax \let\PY@bf=\relax%
    \let\PY@ul=\relax \let\PY@tc=\relax%
    \let\PY@bc=\relax \let\PY@ff=\relax}
\def\PY@tok#1{\csname PY@tok@#1\endcsname}
\def\PY@toks#1+{\ifx\relax#1\empty\else%
    \PY@tok{#1}\expandafter\PY@toks\fi}
\def\PY@do#1{\PY@bc{\PY@tc{\PY@ul{%
    \PY@it{\PY@bf{\PY@ff{#1}}}}}}}
\def\PY#1#2{\PY@reset\PY@toks#1+\relax+\PY@do{#2}}

\@namedef{PY@tok@w}{\def\PY@tc##1{\textcolor[rgb]{0.73,0.73,0.73}{##1}}}
\@namedef{PY@tok@c}{\let\PY@it=\textit\def\PY@tc##1{\textcolor[rgb]{0.24,0.48,0.48}{##1}}}
\@namedef{PY@tok@cp}{\def\PY@tc##1{\textcolor[rgb]{0.61,0.40,0.00}{##1}}}
\@namedef{PY@tok@k}{\let\PY@bf=\textbf\def\PY@tc##1{\textcolor[rgb]{0.00,0.50,0.00}{##1}}}
\@namedef{PY@tok@kp}{\def\PY@tc##1{\textcolor[rgb]{0.00,0.50,0.00}{##1}}}
\@namedef{PY@tok@kt}{\def\PY@tc##1{\textcolor[rgb]{0.69,0.00,0.25}{##1}}}
\@namedef{PY@tok@o}{\def\PY@tc##1{\textcolor[rgb]{0.40,0.40,0.40}{##1}}}
\@namedef{PY@tok@ow}{\let\PY@bf=\textbf\def\PY@tc##1{\textcolor[rgb]{0.67,0.13,1.00}{##1}}}
\@namedef{PY@tok@nb}{\def\PY@tc##1{\textcolor[rgb]{0.00,0.50,0.00}{##1}}}
\@namedef{PY@tok@nf}{\def\PY@tc##1{\textcolor[rgb]{0.00,0.00,1.00}{##1}}}
\@namedef{PY@tok@nc}{\let\PY@bf=\textbf\def\PY@tc##1{\textcolor[rgb]{0.00,0.00,1.00}{##1}}}
\@namedef{PY@tok@nn}{\let\PY@bf=\textbf\def\PY@tc##1{\textcolor[rgb]{0.00,0.00,1.00}{##1}}}
\@namedef{PY@tok@ne}{\let\PY@bf=\textbf\def\PY@tc##1{\textcolor[rgb]{0.80,0.25,0.22}{##1}}}
\@namedef{PY@tok@nv}{\def\PY@tc##1{\textcolor[rgb]{0.10,0.09,0.49}{##1}}}
\@namedef{PY@tok@no}{\def\PY@tc##1{\textcolor[rgb]{0.53,0.00,0.00}{##1}}}
\@namedef{PY@tok@nl}{\def\PY@tc##1{\textcolor[rgb]{0.46,0.46,0.00}{##1}}}
\@namedef{PY@tok@ni}{\let\PY@bf=\textbf\def\PY@tc##1{\textcolor[rgb]{0.44,0.44,0.44}{##1}}}
\@namedef{PY@tok@na}{\def\PY@tc##1{\textcolor[rgb]{0.41,0.47,0.13}{##1}}}
\@namedef{PY@tok@nt}{\let\PY@bf=\textbf\def\PY@tc##1{\textcolor[rgb]{0.00,0.50,0.00}{##1}}}
\@namedef{PY@tok@nd}{\def\PY@tc##1{\textcolor[rgb]{0.67,0.13,1.00}{##1}}}
\@namedef{PY@tok@s}{\def\PY@tc##1{\textcolor[rgb]{0.73,0.13,0.13}{##1}}}
\@namedef{PY@tok@sd}{\let\PY@it=\textit\def\PY@tc##1{\textcolor[rgb]{0.73,0.13,0.13}{##1}}}
\@namedef{PY@tok@si}{\let\PY@bf=\textbf\def\PY@tc##1{\textcolor[rgb]{0.64,0.35,0.47}{##1}}}
\@namedef{PY@tok@se}{\let\PY@bf=\textbf\def\PY@tc##1{\textcolor[rgb]{0.67,0.36,0.12}{##1}}}
\@namedef{PY@tok@sr}{\def\PY@tc##1{\textcolor[rgb]{0.64,0.35,0.47}{##1}}}
\@namedef{PY@tok@ss}{\def\PY@tc##1{\textcolor[rgb]{0.10,0.09,0.49}{##1}}}
\@namedef{PY@tok@sx}{\def\PY@tc##1{\textcolor[rgb]{0.00,0.50,0.00}{##1}}}
\@namedef{PY@tok@m}{\def\PY@tc##1{\textcolor[rgb]{0.40,0.40,0.40}{##1}}}
\@namedef{PY@tok@gh}{\let\PY@bf=\textbf\def\PY@tc##1{\textcolor[rgb]{0.00,0.00,0.50}{##1}}}
\@namedef{PY@tok@gu}{\let\PY@bf=\textbf\def\PY@tc##1{\textcolor[rgb]{0.50,0.00,0.50}{##1}}}
\@namedef{PY@tok@gd}{\def\PY@tc##1{\textcolor[rgb]{0.63,0.00,0.00}{##1}}}
\@namedef{PY@tok@gi}{\def\PY@tc##1{\textcolor[rgb]{0.00,0.52,0.00}{##1}}}
\@namedef{PY@tok@gr}{\def\PY@tc##1{\textcolor[rgb]{0.89,0.00,0.00}{##1}}}
\@namedef{PY@tok@ge}{\let\PY@it=\textit}
\@namedef{PY@tok@gs}{\let\PY@bf=\textbf}
\@namedef{PY@tok@gp}{\let\PY@bf=\textbf\def\PY@tc##1{\textcolor[rgb]{0.00,0.00,0.50}{##1}}}
\@namedef{PY@tok@go}{\def\PY@tc##1{\textcolor[rgb]{0.44,0.44,0.44}{##1}}}
\@namedef{PY@tok@gt}{\def\PY@tc##1{\textcolor[rgb]{0.00,0.27,0.87}{##1}}}
\@namedef{PY@tok@err}{\def\PY@bc##1{{\setlength{\fboxsep}{\string -\fboxrule}\fcolorbox[rgb]{1.00,0.00,0.00}{1,1,1}{\strut ##1}}}}
\@namedef{PY@tok@kc}{\let\PY@bf=\textbf\def\PY@tc##1{\textcolor[rgb]{0.00,0.50,0.00}{##1}}}
\@namedef{PY@tok@kd}{\let\PY@bf=\textbf\def\PY@tc##1{\textcolor[rgb]{0.00,0.50,0.00}{##1}}}
\@namedef{PY@tok@kn}{\let\PY@bf=\textbf\def\PY@tc##1{\textcolor[rgb]{0.00,0.50,0.00}{##1}}}
\@namedef{PY@tok@kr}{\let\PY@bf=\textbf\def\PY@tc##1{\textcolor[rgb]{0.00,0.50,0.00}{##1}}}
\@namedef{PY@tok@bp}{\def\PY@tc##1{\textcolor[rgb]{0.00,0.50,0.00}{##1}}}
\@namedef{PY@tok@fm}{\def\PY@tc##1{\textcolor[rgb]{0.00,0.00,1.00}{##1}}}
\@namedef{PY@tok@vc}{\def\PY@tc##1{\textcolor[rgb]{0.10,0.09,0.49}{##1}}}
\@namedef{PY@tok@vg}{\def\PY@tc##1{\textcolor[rgb]{0.10,0.09,0.49}{##1}}}
\@namedef{PY@tok@vi}{\def\PY@tc##1{\textcolor[rgb]{0.10,0.09,0.49}{##1}}}
\@namedef{PY@tok@vm}{\def\PY@tc##1{\textcolor[rgb]{0.10,0.09,0.49}{##1}}}
\@namedef{PY@tok@sa}{\def\PY@tc##1{\textcolor[rgb]{0.73,0.13,0.13}{##1}}}
\@namedef{PY@tok@sb}{\def\PY@tc##1{\textcolor[rgb]{0.73,0.13,0.13}{##1}}}
\@namedef{PY@tok@sc}{\def\PY@tc##1{\textcolor[rgb]{0.73,0.13,0.13}{##1}}}
\@namedef{PY@tok@dl}{\def\PY@tc##1{\textcolor[rgb]{0.73,0.13,0.13}{##1}}}
\@namedef{PY@tok@s2}{\def\PY@tc##1{\textcolor[rgb]{0.73,0.13,0.13}{##1}}}
\@namedef{PY@tok@sh}{\def\PY@tc##1{\textcolor[rgb]{0.73,0.13,0.13}{##1}}}
\@namedef{PY@tok@s1}{\def\PY@tc##1{\textcolor[rgb]{0.73,0.13,0.13}{##1}}}
\@namedef{PY@tok@mb}{\def\PY@tc##1{\textcolor[rgb]{0.40,0.40,0.40}{##1}}}
\@namedef{PY@tok@mf}{\def\PY@tc##1{\textcolor[rgb]{0.40,0.40,0.40}{##1}}}
\@namedef{PY@tok@mh}{\def\PY@tc##1{\textcolor[rgb]{0.40,0.40,0.40}{##1}}}
\@namedef{PY@tok@mi}{\def\PY@tc##1{\textcolor[rgb]{0.40,0.40,0.40}{##1}}}
\@namedef{PY@tok@il}{\def\PY@tc##1{\textcolor[rgb]{0.40,0.40,0.40}{##1}}}
\@namedef{PY@tok@mo}{\def\PY@tc##1{\textcolor[rgb]{0.40,0.40,0.40}{##1}}}
\@namedef{PY@tok@ch}{\let\PY@it=\textit\def\PY@tc##1{\textcolor[rgb]{0.24,0.48,0.48}{##1}}}
\@namedef{PY@tok@cm}{\let\PY@it=\textit\def\PY@tc##1{\textcolor[rgb]{0.24,0.48,0.48}{##1}}}
\@namedef{PY@tok@cpf}{\let\PY@it=\textit\def\PY@tc##1{\textcolor[rgb]{0.24,0.48,0.48}{##1}}}
\@namedef{PY@tok@c1}{\let\PY@it=\textit\def\PY@tc##1{\textcolor[rgb]{0.24,0.48,0.48}{##1}}}
\@namedef{PY@tok@cs}{\let\PY@it=\textit\def\PY@tc##1{\textcolor[rgb]{0.24,0.48,0.48}{##1}}}

\def\PYZbs{\char`\\}
\def\PYZus{\char`\_}
\def\PYZob{\char`\{}
\def\PYZcb{\char`\}}
\def\PYZca{\char`\^}
\def\PYZam{\char`\&}
\def\PYZlt{\char`\<}
\def\PYZgt{\char`\>}
\def\PYZsh{\char`\#}
\def\PYZpc{\char`\%}
\def\PYZdl{\char`\$}
\def\PYZhy{\char`\-}
\def\PYZsq{\char`\'}
\def\PYZdq{\char`\"}
\def\PYZti{\char`\~}
% for compatibility with earlier versions
\def\PYZat{@}
\def\PYZlb{[}
\def\PYZrb{]}
\makeatother


    % For linebreaks inside Verbatim environment from package fancyvrb.
    \makeatletter
        \newbox\Wrappedcontinuationbox
        \newbox\Wrappedvisiblespacebox
        \newcommand*\Wrappedvisiblespace {\textcolor{red}{\textvisiblespace}}
        \newcommand*\Wrappedcontinuationsymbol {\textcolor{red}{\llap{\tiny$\m@th\hookrightarrow$}}}
        \newcommand*\Wrappedcontinuationindent {3ex }
        \newcommand*\Wrappedafterbreak {\kern\Wrappedcontinuationindent\copy\Wrappedcontinuationbox}
        % Take advantage of the already applied Pygments mark-up to insert
        % potential linebreaks for TeX processing.
        %        {, <, #, %, $, ' and ": go to next line.
        %        _, }, ^, &, >, - and ~: stay at end of broken line.
        % Use of \textquotesingle for straight quote.
        \newcommand*\Wrappedbreaksatspecials {%
            \def\PYGZus{\discretionary{\char`\_}{\Wrappedafterbreak}{\char`\_}}%
            \def\PYGZob{\discretionary{}{\Wrappedafterbreak\char`\{}{\char`\{}}%
            \def\PYGZcb{\discretionary{\char`\}}{\Wrappedafterbreak}{\char`\}}}%
            \def\PYGZca{\discretionary{\char`\^}{\Wrappedafterbreak}{\char`\^}}%
            \def\PYGZam{\discretionary{\char`\&}{\Wrappedafterbreak}{\char`\&}}%
            \def\PYGZlt{\discretionary{}{\Wrappedafterbreak\char`\<}{\char`\<}}%
            \def\PYGZgt{\discretionary{\char`\>}{\Wrappedafterbreak}{\char`\>}}%
            \def\PYGZsh{\discretionary{}{\Wrappedafterbreak\char`\#}{\char`\#}}%
            \def\PYGZpc{\discretionary{}{\Wrappedafterbreak\char`\%}{\char`\%}}%
            \def\PYGZdl{\discretionary{}{\Wrappedafterbreak\char`\$}{\char`\$}}%
            \def\PYGZhy{\discretionary{\char`\-}{\Wrappedafterbreak}{\char`\-}}%
            \def\PYGZsq{\discretionary{}{\Wrappedafterbreak\textquotesingle}{\textquotesingle}}%
            \def\PYGZdq{\discretionary{}{\Wrappedafterbreak\char`\"}{\char`\"}}%
            \def\PYGZti{\discretionary{\char`\~}{\Wrappedafterbreak}{\char`\~}}%
        }
        % Some characters . , ; ? ! / are not pygmentized.
        % This macro makes them "active" and they will insert potential linebreaks
        \newcommand*\Wrappedbreaksatpunct {%
            \lccode`\~`\.\lowercase{\def~}{\discretionary{\hbox{\char`\.}}{\Wrappedafterbreak}{\hbox{\char`\.}}}%
            \lccode`\~`\,\lowercase{\def~}{\discretionary{\hbox{\char`\,}}{\Wrappedafterbreak}{\hbox{\char`\,}}}%
            \lccode`\~`\;\lowercase{\def~}{\discretionary{\hbox{\char`\;}}{\Wrappedafterbreak}{\hbox{\char`\;}}}%
            \lccode`\~`\:\lowercase{\def~}{\discretionary{\hbox{\char`\:}}{\Wrappedafterbreak}{\hbox{\char`\:}}}%
            \lccode`\~`\?\lowercase{\def~}{\discretionary{\hbox{\char`\?}}{\Wrappedafterbreak}{\hbox{\char`\?}}}%
            \lccode`\~`\!\lowercase{\def~}{\discretionary{\hbox{\char`\!}}{\Wrappedafterbreak}{\hbox{\char`\!}}}%
            \lccode`\~`\/\lowercase{\def~}{\discretionary{\hbox{\char`\/}}{\Wrappedafterbreak}{\hbox{\char`\/}}}%
            \catcode`\.\active
            \catcode`\,\active
            \catcode`\;\active
            \catcode`\:\active
            \catcode`\?\active
            \catcode`\!\active
            \catcode`\/\active
            \lccode`\~`\~
        }
    \makeatother

    \let\OriginalVerbatim=\Verbatim
    \makeatletter
    \renewcommand{\Verbatim}[1][1]{%
        %\parskip\z@skip
        \sbox\Wrappedcontinuationbox {\Wrappedcontinuationsymbol}%
        \sbox\Wrappedvisiblespacebox {\FV@SetupFont\Wrappedvisiblespace}%
        \def\FancyVerbFormatLine ##1{\hsize\linewidth
            \vtop{\raggedright\hyphenpenalty\z@\exhyphenpenalty\z@
                \doublehyphendemerits\z@\finalhyphendemerits\z@
                \strut ##1\strut}%
        }%
        % If the linebreak is at a space, the latter will be displayed as visible
        % space at end of first line, and a continuation symbol starts next line.
        % Stretch/shrink are however usually zero for typewriter font.
        \def\FV@Space {%
            \nobreak\hskip\z@ plus\fontdimen3\font minus\fontdimen4\font
            \discretionary{\copy\Wrappedvisiblespacebox}{\Wrappedafterbreak}
            {\kern\fontdimen2\font}%
        }%

        % Allow breaks at special characters using \PYG... macros.
        \Wrappedbreaksatspecials
        % Breaks at punctuation characters . , ; ? ! and / need catcode=\active
        \OriginalVerbatim[#1,codes*=\Wrappedbreaksatpunct]%
    }
    \makeatother

    % Exact colors from NB
    \definecolor{incolor}{HTML}{303F9F}
    \definecolor{outcolor}{HTML}{D84315}
    \definecolor{cellborder}{HTML}{CFCFCF}
    \definecolor{cellbackground}{HTML}{F7F7F7}

    % prompt
    \makeatletter
    \newcommand{\boxspacing}{\kern\kvtcb@left@rule\kern\kvtcb@boxsep}
    \makeatother
    \newcommand{\prompt}[4]{
        {\ttfamily\llap{{\color{#2}[#3]:\hspace{3pt}#4}}\vspace{-\baselineskip}}
    }
    

    
    % Prevent overflowing lines due to hard-to-break entities
    \sloppy
    % Setup hyperref package
    \hypersetup{
      breaklinks=true,  % so long urls are correctly broken across lines
      colorlinks=true,
      urlcolor=urlcolor,
      linkcolor=linkcolor,
      citecolor=citecolor,
      }
    % Slightly bigger margins than the latex defaults
    
    \geometry{verbose,tmargin=1in,bmargin=1in,lmargin=1in,rmargin=1in}
    
    

\begin{document}
    
    \maketitle
    
    

    
    \hypertarget{ux43bux430ux431ux43eux440ux430ux442ux43eux440ux43dux430ux44f-ux440ux430ux431ux43eux442ux430-1-ux432ux432ux435ux434ux435ux43dux438ux435-ux432-ux430ux43dux430ux43bux438ux437-ux434ux430ux43dux43dux44bux445-ux441-ux43fux43eux43cux43eux449ux44cux44e-python}{%
\section{Лабораторная работа №1, Введение в анализ данных с помощью
Python}\label{ux43bux430ux431ux43eux440ux430ux442ux43eux440ux43dux430ux44f-ux440ux430ux431ux43eux442ux430-1-ux432ux432ux435ux434ux435ux43dux438ux435-ux432-ux430ux43dux430ux43bux438ux437-ux434ux430ux43dux43dux44bux445-ux441-ux43fux43eux43cux43eux449ux44cux44e-python}}

\hypertarget{ux444ux438ux43e-ux436ux438ux434ux43aux43eux432-ux430ux440ux442ux435ux43cux438ux439-ux430ux43dux434ux440ux435ux435ux432ux438ux447}{%
\section{ФИО: Жидков Артемий
Андреевич}\label{ux444ux438ux43e-ux436ux438ux434ux43aux43eux432-ux430ux440ux442ux435ux43cux438ux439-ux430ux43dux434ux440ux435ux435ux432ux438ux447}}

\hypertarget{ux433ux440ux443ux43fux43fux430-r4136ux441}{%
\section{группа:
R4136с}\label{ux433ux440ux443ux43fux43fux430-r4136ux441}}

    \hypertarget{ux438ux43cux43fux43eux440ux442ux438ux440ux43eux432ux430ux442ux44c-ux431ux438ux431ux43bux438ux43eux442ux435ux43aux438-ux432-python.}{%
\subsubsection{1. Импортировать библиотеки в
Python.}\label{ux438ux43cux43fux43eux440ux442ux438ux440ux43eux432ux430ux442ux44c-ux431ux438ux431ux43bux438ux43eux442ux435ux43aux438-ux432-python.}}

    \begin{tcolorbox}[breakable, size=fbox, boxrule=1pt, pad at break*=1mm,colback=cellbackground, colframe=cellborder]
\prompt{In}{incolor}{84}{\boxspacing}
\begin{Verbatim}[commandchars=\\\{\}]
\PY{k+kn}{import} \PY{n+nn}{numpy} \PY{k}{as} \PY{n+nn}{np}
\PY{k+kn}{import} \PY{n+nn}{matplotlib}\PY{n+nn}{.}\PY{n+nn}{pyplot} \PY{k}{as} \PY{n+nn}{plt}
\PY{k+kn}{import} \PY{n+nn}{os}
\PY{k+kn}{import} \PY{n+nn}{random}
\PY{k+kn}{import} \PY{n+nn}{scipy}
\PY{k+kn}{import} \PY{n+nn}{torch}

\PY{n}{torch}\PY{o}{.}\PY{n}{cuda}\PY{o}{.}\PY{n}{synchronize}\PY{p}{(}\PY{p}{)}
\PY{n}{torch}\PY{o}{.}\PY{n}{cuda}\PY{o}{.}\PY{n}{empty\PYZus{}cache}\PY{p}{(}\PY{p}{)}

\PY{n}{cuda} \PY{o}{=} \PY{n}{torch}\PY{o}{.}\PY{n}{device}\PY{p}{(}\PY{l+s+s1}{\PYZsq{}}\PY{l+s+s1}{cuda}\PY{l+s+s1}{\PYZsq{}}\PY{p}{)}
\PY{n+nb}{print}\PY{p}{(}\PY{n}{torch}\PY{o}{.}\PY{n}{cuda}\PY{o}{.}\PY{n}{get\PYZus{}device\PYZus{}properties}\PY{p}{(}\PY{n}{cuda}\PY{p}{)}\PY{p}{)}
\end{Verbatim}
\end{tcolorbox}

    \begin{Verbatim}[commandchars=\\\{\}]
\_CudaDeviceProperties(name='NVIDIA GeForce RTX 3080 Laptop GPU', major=8,
minor=6, total\_memory=8191MB, multi\_processor\_count=48)
    \end{Verbatim}

    \hypertarget{ux437ux430ux433ux440ux443ux437ux43aux430-ux438-ux43fux43eux434ux433ux43eux442ux43eux432ux43aux430-ux434ux430ux43dux43dux44bux445.}{%
\subsubsection{2. Загрузка и подготовка
данных.}\label{ux437ux430ux433ux440ux443ux437ux43aux430-ux438-ux43fux43eux434ux433ux43eux442ux43eux432ux43aux430-ux434ux430ux43dux43dux44bux445.}}

    \begin{tcolorbox}[breakable, size=fbox, boxrule=1pt, pad at break*=1mm,colback=cellbackground, colframe=cellborder]
\prompt{In}{incolor}{85}{\boxspacing}
\begin{Verbatim}[commandchars=\\\{\}]
\PY{n}{name} \PY{o}{=} \PY{n}{random}\PY{o}{.}\PY{n}{choice}\PY{p}{(}\PY{n}{os}\PY{o}{.}\PY{n}{listdir}\PY{p}{(}\PY{l+s+s2}{\PYZdq{}}\PY{l+s+s2}{dataset}\PY{l+s+s2}{\PYZdq{}}\PY{p}{)}\PY{p}{)}

\PY{c+c1}{\PYZsh{} name = \PYZsq{}testLab1Var7.csv\PYZsq{}}

\PY{n+nb}{print}\PY{p}{(}\PY{l+s+sa}{f}\PY{l+s+s2}{\PYZdq{}}\PY{l+s+s2}{Dataset: }\PY{l+s+si}{\PYZob{}}\PY{n}{name}\PY{l+s+si}{\PYZcb{}}\PY{l+s+s2}{\PYZdq{}}\PY{p}{)}

\PY{n}{dataset} \PY{o}{=} \PY{n}{np}\PY{o}{.}\PY{n}{genfromtxt}\PY{p}{(}\PY{l+s+sa}{f}\PY{l+s+s2}{\PYZdq{}}\PY{l+s+s2}{dataset/}\PY{l+s+si}{\PYZob{}}\PY{n}{name}\PY{l+s+si}{\PYZcb{}}\PY{l+s+s2}{\PYZdq{}}\PY{p}{,} \PY{n}{delimiter}\PY{o}{=}\PY{l+s+s1}{\PYZsq{}}\PY{l+s+s1}{,}\PY{l+s+s1}{\PYZsq{}}\PY{p}{)}

\PY{n}{dataset} \PY{o}{=} \PY{p}{[}\PY{n}{dataset}\PY{p}{[}\PY{p}{:}\PY{p}{,} \PY{n}{i}\PY{p}{]} \PY{k}{for} \PY{n}{i} \PY{o+ow}{in} \PY{n+nb}{range}\PY{p}{(}\PY{n}{dataset}\PY{o}{.}\PY{n}{shape}\PY{p}{[}\PY{l+m+mi}{1}\PY{p}{]}\PY{p}{)}\PY{p}{]}
\PY{n}{title} \PY{o}{=} \PY{p}{[}\PY{l+s+s2}{\PYZdq{}}\PY{l+s+s2}{time}\PY{l+s+s2}{\PYZdq{}}\PY{p}{,} \PY{l+s+s2}{\PYZdq{}}\PY{l+s+s2}{current}\PY{l+s+s2}{\PYZdq{}}\PY{p}{,} \PY{l+s+s2}{\PYZdq{}}\PY{l+s+s2}{voltage}\PY{l+s+s2}{\PYZdq{}}\PY{p}{]}

\PY{n}{dataset\PYZus{}dict} \PY{o}{=} \PY{n+nb}{dict}\PY{p}{(}\PY{n+nb}{zip}\PY{p}{(}\PY{n}{title}\PY{p}{,} \PY{n}{dataset}\PY{p}{)}\PY{p}{)}
\end{Verbatim}
\end{tcolorbox}

    \begin{Verbatim}[commandchars=\\\{\}]
Dataset: testLab1Var56.csv
    \end{Verbatim}

    \hypertarget{ux43dux430ux440ux438ux441ux43eux432ux430ux442ux44c-ux433ux440ux430ux444ux438ux43aux438-ux442ux43eux43aux430-ux438-ux43dux430ux43fux440ux44fux436ux435ux43dux438ux44f.}{%
\subsubsection{3. Нарисовать графики тока и
напряжения.}\label{ux43dux430ux440ux438ux441ux43eux432ux430ux442ux44c-ux433ux440ux430ux444ux438ux43aux438-ux442ux43eux43aux430-ux438-ux43dux430ux43fux440ux44fux436ux435ux43dux438ux44f.}}

Для удобства отображения отображу не весь график, а некоторый его
случайный диапазон заданного размера, установив лимиты на данные.

    \begin{tcolorbox}[breakable, size=fbox, boxrule=1pt, pad at break*=1mm,colback=cellbackground, colframe=cellborder]
\prompt{In}{incolor}{86}{\boxspacing}
\begin{Verbatim}[commandchars=\\\{\}]
\PY{l+s+sd}{\PYZdq{}\PYZdq{}\PYZdq{}}
\PY{l+s+sd}{Размер интервала}
\PY{l+s+sd}{\PYZdq{}\PYZdq{}\PYZdq{}}
\PY{n}{time\PYZus{}period} \PY{o}{=} \PY{l+m+mf}{0.1}
\end{Verbatim}
\end{tcolorbox}

    \begin{tcolorbox}[breakable, size=fbox, boxrule=1pt, pad at break*=1mm,colback=cellbackground, colframe=cellborder]
\prompt{In}{incolor}{87}{\boxspacing}
\begin{Verbatim}[commandchars=\\\{\}]
\PY{n}{time\PYZus{}interval} \PY{o}{=} \PY{n}{random}\PY{o}{.}\PY{n}{random}\PY{p}{(}\PY{p}{)} \PY{o}{*} \PY{p}{(}\PY{n}{dataset\PYZus{}dict}\PY{p}{[}\PY{l+s+s2}{\PYZdq{}}\PY{l+s+s2}{time}\PY{l+s+s2}{\PYZdq{}}\PY{p}{]}\PY{p}{[}\PY{o}{\PYZhy{}}\PY{l+m+mi}{1}\PY{p}{]} \PY{o}{\PYZhy{}} \PY{n}{time\PYZus{}period}\PY{p}{)}
\PY{n}{time\PYZus{}interval} \PY{o}{=} \PY{p}{(}\PY{n}{time\PYZus{}interval}\PY{p}{,} \PY{n}{time\PYZus{}interval} \PY{o}{+} \PY{n}{time\PYZus{}period}\PY{p}{)}

\PY{n+nb}{print}\PY{p}{(}\PY{l+s+sa}{f}\PY{l+s+s2}{\PYZdq{}}\PY{l+s+s2}{Временной интервал }\PY{l+s+si}{\PYZob{}}\PY{n}{time\PYZus{}interval}\PY{l+s+si}{\PYZcb{}}\PY{l+s+s2}{\PYZdq{}}\PY{p}{)}
\end{Verbatim}
\end{tcolorbox}

    \begin{Verbatim}[commandchars=\\\{\}]
Временной интервал (42.89388613861019, 42.99388613861019)
    \end{Verbatim}

    \begin{tcolorbox}[breakable, size=fbox, boxrule=1pt, pad at break*=1mm,colback=cellbackground, colframe=cellborder]
\prompt{In}{incolor}{88}{\boxspacing}
\begin{Verbatim}[commandchars=\\\{\}]
\PY{n}{plt}\PY{o}{.}\PY{n}{plot}\PY{p}{(}\PY{n}{dataset\PYZus{}dict}\PY{p}{[}\PY{l+s+s2}{\PYZdq{}}\PY{l+s+s2}{time}\PY{l+s+s2}{\PYZdq{}}\PY{p}{]}\PY{p}{,} \PY{n}{dataset\PYZus{}dict}\PY{p}{[}\PY{l+s+s2}{\PYZdq{}}\PY{l+s+s2}{current}\PY{l+s+s2}{\PYZdq{}}\PY{p}{]}\PY{p}{)}
\PY{n}{plt}\PY{o}{.}\PY{n}{xlim}\PY{p}{(}\PY{n}{time\PYZus{}interval}\PY{p}{)}
\PY{n}{plt}\PY{o}{.}\PY{n}{xlabel}\PY{p}{(}\PY{l+s+s1}{\PYZsq{}}\PY{l+s+s1}{Время, с}\PY{l+s+s1}{\PYZsq{}}\PY{p}{)}
\PY{n}{plt}\PY{o}{.}\PY{n}{ylabel}\PY{p}{(}\PY{l+s+s1}{\PYZsq{}}\PY{l+s+s1}{Сила Тока, А}\PY{l+s+s1}{\PYZsq{}}\PY{p}{)}
\PY{n}{plt}\PY{o}{.}\PY{n}{legend}\PY{p}{(}\PY{p}{[}\PY{l+s+s2}{\PYZdq{}}\PY{l+s+s2}{Сила Тока в моторе}\PY{l+s+s2}{\PYZdq{}}\PY{p}{]}\PY{p}{)}
\PY{n}{plt}\PY{o}{.}\PY{n}{grid}\PY{p}{(}\PY{p}{)}
\end{Verbatim}
\end{tcolorbox}

    \begin{center}
    \adjustimage{max size={0.9\linewidth}{0.9\paperheight}}{lab1_files/lab1_8_0.png}
    \end{center}
    { \hspace*{\fill} \\}
    
    \begin{tcolorbox}[breakable, size=fbox, boxrule=1pt, pad at break*=1mm,colback=cellbackground, colframe=cellborder]
\prompt{In}{incolor}{89}{\boxspacing}
\begin{Verbatim}[commandchars=\\\{\}]
\PY{n}{plt}\PY{o}{.}\PY{n}{plot}\PY{p}{(}\PY{n}{dataset\PYZus{}dict}\PY{p}{[}\PY{l+s+s2}{\PYZdq{}}\PY{l+s+s2}{time}\PY{l+s+s2}{\PYZdq{}}\PY{p}{]}\PY{p}{,} \PY{n}{dataset\PYZus{}dict}\PY{p}{[}\PY{l+s+s2}{\PYZdq{}}\PY{l+s+s2}{voltage}\PY{l+s+s2}{\PYZdq{}}\PY{p}{]}\PY{p}{)}
\PY{n}{plt}\PY{o}{.}\PY{n}{xlim}\PY{p}{(}\PY{n}{time\PYZus{}interval}\PY{p}{)}
\PY{n}{plt}\PY{o}{.}\PY{n}{xlabel}\PY{p}{(}\PY{l+s+s1}{\PYZsq{}}\PY{l+s+s1}{Время, с}\PY{l+s+s1}{\PYZsq{}}\PY{p}{)}
\PY{n}{plt}\PY{o}{.}\PY{n}{ylabel}\PY{p}{(}\PY{l+s+s1}{\PYZsq{}}\PY{l+s+s1}{Напряжение, В}\PY{l+s+s1}{\PYZsq{}}\PY{p}{)}
\PY{n}{plt}\PY{o}{.}\PY{n}{legend}\PY{p}{(}\PY{p}{[}\PY{l+s+s2}{\PYZdq{}}\PY{l+s+s2}{Напряжение на моторе}\PY{l+s+s2}{\PYZdq{}}\PY{p}{]}\PY{p}{)}
\PY{n}{plt}\PY{o}{.}\PY{n}{grid}\PY{p}{(}\PY{p}{)}
\end{Verbatim}
\end{tcolorbox}

    \begin{center}
    \adjustimage{max size={0.9\linewidth}{0.9\paperheight}}{lab1_files/lab1_9_0.png}
    \end{center}
    { \hspace*{\fill} \\}
    
    \hypertarget{ux440ux430ux441ux441ux447ux438ux442ux430ux442ux44c-ux437ux43dux430ux447ux435ux43dux438ux44f-ux43fux430ux440ux430ux43cux435ux442ux440ux43eux432-l-ux438-r.}{%
\subsubsection{4. Рассчитать значения параметров L и
R.}\label{ux440ux430ux441ux441ux447ux438ux442ux430ux442ux44c-ux437ux43dux430ux447ux435ux43dux438ux44f-ux43fux430ux440ux430ux43cux435ux442ux440ux43eux432-l-ux438-r.}}

    \href{https://life-prog.ru/2_22546_model-dvigatelya-postoyannogo-toka-s-nezavisimim-vozbuzhdeniem.html}{Упрощённая
модель двигателя постоянного тока}. Модель двигателя постоянного тока
описывается следующей системой дифференциальных уравнений:

\[
\begin{cases}
u = e + R \times i + L \times \dfrac{di}{t}
\\
M - M_C = J \dfrac{d\omega}{t}
\\
M = C_{M} \times \Phi \times i
\\
e = C_{\omega} \times \Phi \times \omega
\end{cases}
\]

где

\(u\) - напряжение на якорной обмотке двигателя,

\(e\) - электродвижущая сила (ЭДС) якоря,

\(i\) - ток якоря,

\(\Phi\) - поток, создаваемый обмоткой возбуждения,

\(M\) - электромагнитный момент двигателя,

\(M_С\) - момент сопротивления движению,

\(\omega\) - скорость вращения вала двигателя,

\(R\) - активное сопротивление якорной цепи,

\(L\) - индуктивность якорной цепи,

\(J\) - суммарный момент инерции якоря и нагрузки,

\(С_{\omega}\) - коэффициент связи между скоростью и ЭДС,

\(С_М\) - коэффициент связи между током якоря и электромагнитным
моментом.

    Согласно упрощению в данной работе двигатель застопорен, следовательно:

\[
\omega = 0 \to e = C_{\omega} \times \Phi \times \omega = 0 \to u = R \times i + L \times \dfrac{di}{t}
\]

Тогда можно получить уравнение электрической составляющей двигателя:

\[
L \times \dfrac{di}{t} = u  - R \times i
\]

\[
\dfrac{di}{t} = \dfrac{u}{L}  - \dfrac{R}{L} \times i
\]

Что в операторном виде \(s = \dfrac{d}{t}\) будет иметь вид:

\[
s \times i = \dfrac{u}{L}  - \dfrac{R}{L} \times i
\]

Теперь нужно составить передаточную функцию, выразив соотношение
\(G_c = \dfrac{i}{u}\):

\[
G_c(s) = \dfrac{1}{L \times (s + \dfrac{R}{L})}
\]

Применив
\href{https://x-engineer.org/discretizing-transfer-function/\#:~:text=Discretization\%20is\%20the\%20process\%20through,out\%20computing\%20and\%20control\%20tasks.}{Forward
Euler (difference) discretization}, получу:

\[
s = \dfrac{z - 1}{T_d}
\]

\[
G_d(z) = G_c(s=\dfrac{z - 1}{T_s}) = \dfrac{1}{L \times (\dfrac{z - 1}{T_s} + \dfrac{R}{L})}
\]

\[
G_d(z_i=z^{-1}) = \dfrac{T_d}{R \times T_d - L + L \times z_i^{-1}}
\]

\[
G_d(z_i) = \dfrac{T_d \times z_i}{L - L \times z_i + R \times T_d \times z_i}
\]

Данное выражение нужно выразить относительно констант и переменной
\(z^{-1}\):

\[
G_d(z=z_i^{-1}) = \dfrac{T_d \times z^{-1}}{L - L \times z^{-1} + R \times T_d \times z^{-1}} = \dfrac{i(z)}{u(z)}
\]

\[
T_d \times u(z)*z^{-1} = i(z)*(L - L \times z^{-1} + R \times T_d \times z^{-1})
\]

\[
T_d \times u(z)*z^{-1} = i(z)*L - i(z)*L \times z^{-1} + i(z)*R \times T_d \times z^{-1}
\]

\[
i(z)*L = T_d \times u(z)*z^{-1} + i(z)*L \times z^{-1} - i(z)*R \times T_d \times z^{-1}
\]

\[
i(z) = u(z)*z^{-1} \times (\dfrac{T_d}{L}) + i(z)*z^{-1} \times (1 - \dfrac{R \times T_d}{L})
\]

\[
i(z) = u(z)*z^{-1} \times (\dfrac{T_d}{L}) - i(z)*z^{-1} \times (\dfrac{R \times T_d - L}{L})
\]

В дискретной системе операция \(z^{-1}\) - есть задержка на один такт,
следовательно, можно рассчитать значение силы тока в конкретны момент:

\[
i_{k} = u_{k-1} \times (\dfrac{T_d}{L}) - i_{k-1} \times (\dfrac{R \times T_d - L}{L})
\]

    Для упрощения задачи уже линейной аппроксимации функции введу матрицы
для хранения и более удобного представления:

\[
Y_{n \times 1} = i_{1:end}
\]

\[
X_{n \times 2} = [ i_{0:end-1} | u_{0:end-1} ]
\]

\[
K_{2 \times 1} = [\dfrac{T_d}{L} | \dfrac{R \times T_d - L}{L}]^{T}
\]

\[
Y = X * K
\]

Следующий код подготавливает нужные матрицы и загружает их в память
видеокарты:

    

    \begin{tcolorbox}[breakable, size=fbox, boxrule=1pt, pad at break*=1mm,colback=cellbackground, colframe=cellborder]
\prompt{In}{incolor}{90}{\boxspacing}
\begin{Verbatim}[commandchars=\\\{\}]
\PY{n}{X} \PY{o}{=} \PY{n}{np}\PY{o}{.}\PY{n}{transpose}\PY{p}{(}\PY{n}{np}\PY{o}{.}\PY{n}{concatenate}\PY{p}{(}\PY{p}{[}\PY{n}{np}\PY{o}{.}\PY{n}{array}\PY{p}{(}\PY{p}{[}\PY{n}{dataset\PYZus{}dict}\PY{p}{[}\PY{l+s+s2}{\PYZdq{}}\PY{l+s+s2}{voltage}\PY{l+s+s2}{\PYZdq{}}\PY{p}{]}\PY{p}{,} \PY{p}{]}\PY{p}{)}\PY{p}{,} \PY{n}{np}\PY{o}{.}\PY{n}{array}\PY{p}{(}\PY{p}{[}\PY{n}{dataset\PYZus{}dict}\PY{p}{[}\PY{l+s+s2}{\PYZdq{}}\PY{l+s+s2}{current}\PY{l+s+s2}{\PYZdq{}}\PY{p}{]}\PY{p}{,} \PY{p}{]}\PY{p}{)}\PY{p}{]}\PY{p}{,} \PY{n}{axis}\PY{o}{=}\PY{l+m+mi}{0}\PY{p}{)}\PY{p}{)}
\PY{n}{Y} \PY{o}{=} \PY{n}{np}\PY{o}{.}\PY{n}{transpose}\PY{p}{(}\PY{n}{np}\PY{o}{.}\PY{n}{array}\PY{p}{(}\PY{p}{[}\PY{n}{dataset\PYZus{}dict}\PY{p}{[}\PY{l+s+s2}{\PYZdq{}}\PY{l+s+s2}{current}\PY{l+s+s2}{\PYZdq{}}\PY{p}{]}\PY{p}{,} \PY{p}{]}\PY{p}{)}\PY{p}{)}

\PY{c+c1}{\PYZsh{} X : n*k}
\PY{c+c1}{\PYZsh{} K : k*1}
\PY{c+c1}{\PYZsh{} X * K = Y}
\PY{c+c1}{\PYZsh{} [U(k\PYZhy{}1);I(k\PYZhy{}1)] * K = [I(k)]}

\PY{n}{X} \PY{o}{=} \PY{n}{X}\PY{p}{[}\PY{p}{:}\PY{o}{\PYZhy{}}\PY{l+m+mi}{1}\PY{p}{,} \PY{p}{:}\PY{p}{]}  \PY{c+c1}{\PYZsh{} U(k\PYZhy{}1);I(k\PYZhy{}1)}
\PY{n}{Y} \PY{o}{=} \PY{n}{Y}\PY{p}{[}\PY{l+m+mi}{1}\PY{p}{:}\PY{p}{,} \PY{p}{:}\PY{p}{]}  \PY{c+c1}{\PYZsh{} I(k)}

\PY{n+nb}{print}\PY{p}{(}\PY{n}{X}\PY{o}{.}\PY{n}{shape}\PY{p}{)}
\PY{n+nb}{print}\PY{p}{(}\PY{n}{Y}\PY{o}{.}\PY{n}{shape}\PY{p}{)}

\PY{n}{X\PYZus{}tensor} \PY{o}{=} \PY{n}{torch}\PY{o}{.}\PY{n}{tensor}\PY{p}{(}\PY{n}{X}\PY{p}{,} \PY{n}{device}\PY{o}{=}\PY{n}{cuda}\PY{p}{,} \PY{n}{dtype}\PY{o}{=}\PY{n}{torch}\PY{o}{.}\PY{n}{float64}\PY{p}{)}
\PY{n}{Y\PYZus{}tensor} \PY{o}{=} \PY{n}{torch}\PY{o}{.}\PY{n}{tensor}\PY{p}{(}\PY{n}{Y}\PY{p}{,} \PY{n}{device}\PY{o}{=}\PY{n}{cuda}\PY{p}{,} \PY{n}{dtype}\PY{o}{=}\PY{n}{torch}\PY{o}{.}\PY{n}{float64}\PY{p}{)}
\end{Verbatim}
\end{tcolorbox}

    \begin{Verbatim}[commandchars=\\\{\}]
(100000, 2)
(100000, 1)
    \end{Verbatim}

    Функция нахождения
\href{https://en.wikipedia.org/wiki/Moore\%E2\%80\%93Penrose_inverse}{псевдообратной
матрицы Moore--Penrose} через Сингулярное Разложение.

Данная матрица позволяет аппроксимировать \(Y = X \times K + e\)
\href{https://www.gaussianwaves.com/2014/06/linear-models-least-squares-estimator-lse/amp/}{Методом
Наименьших Квадратов}:

    \begin{tcolorbox}[breakable, size=fbox, boxrule=1pt, pad at break*=1mm,colback=cellbackground, colframe=cellborder]
\prompt{In}{incolor}{91}{\boxspacing}
\begin{Verbatim}[commandchars=\\\{\}]
\PY{k}{def} \PY{n+nf}{get\PYZus{}pseudoinverse}\PY{p}{(}\PY{n}{matrix}\PY{p}{)}\PY{p}{:}
    \PY{n}{matrix\PYZus{}svd} \PY{o}{=} \PY{n}{torch}\PY{o}{.}\PY{n}{svd}\PY{p}{(}\PY{n}{matrix}\PY{p}{)}

    \PY{n}{matrix\PYZus{}psi} \PY{o}{=} \PY{n}{matrix\PYZus{}svd}\PY{o}{.}\PY{n}{V}
    \PY{n}{matrix\PYZus{}psi} \PY{o}{=} \PY{n}{torch}\PY{o}{.}\PY{n}{mm}\PY{p}{(}\PY{n}{matrix\PYZus{}psi}\PY{p}{,} \PY{n}{torch}\PY{o}{.}\PY{n}{diag}\PY{p}{(}\PY{l+m+mi}{1} \PY{o}{/} \PY{n}{matrix\PYZus{}svd}\PY{o}{.}\PY{n}{S}\PY{p}{)}\PY{p}{)}
    \PY{n}{matrix\PYZus{}psi} \PY{o}{=} \PY{n}{torch}\PY{o}{.}\PY{n}{mm}\PY{p}{(}\PY{n}{matrix\PYZus{}psi}\PY{p}{,} \PY{n}{matrix\PYZus{}svd}\PY{o}{.}\PY{n}{U}\PY{o}{.}\PY{n}{T}\PY{p}{)}

    \PY{k}{return} \PY{n}{matrix\PYZus{}psi}
\end{Verbatim}
\end{tcolorbox}

    \begin{tcolorbox}[breakable, size=fbox, boxrule=1pt, pad at break*=1mm,colback=cellbackground, colframe=cellborder]
\prompt{In}{incolor}{92}{\boxspacing}
\begin{Verbatim}[commandchars=\\\{\}]
\PY{n}{X\PYZus{}psi} \PY{o}{=} \PY{n}{get\PYZus{}pseudoinverse}\PY{p}{(}\PY{n}{X\PYZus{}tensor}\PY{p}{)}

\PY{n+nb}{print}\PY{p}{(}\PY{n}{X\PYZus{}psi}\PY{o}{.}\PY{n}{shape}\PY{p}{)}
\PY{n+nb}{print}\PY{p}{(}\PY{n}{X\PYZus{}tensor}\PY{o}{.}\PY{n}{shape}\PY{p}{)}
\end{Verbatim}
\end{tcolorbox}

    \begin{Verbatim}[commandchars=\\\{\}]
torch.Size([2, 100000])
torch.Size([100000, 2])
    \end{Verbatim}

    Результатом перемножения псевдообратной матрицы от X и самой матрицы X
должна быть матрица, близкая к единичной:

    \begin{tcolorbox}[breakable, size=fbox, boxrule=1pt, pad at break*=1mm,colback=cellbackground, colframe=cellborder]
\prompt{In}{incolor}{93}{\boxspacing}
\begin{Verbatim}[commandchars=\\\{\}]
\PY{n+nb}{print}\PY{p}{(}\PY{n}{torch}\PY{o}{.}\PY{n}{mm}\PY{p}{(}\PY{n}{X\PYZus{}psi}\PY{p}{,} \PY{n}{X\PYZus{}tensor}\PY{p}{)}\PY{p}{)}
\end{Verbatim}
\end{tcolorbox}

    \begin{Verbatim}[commandchars=\\\{\}]
tensor([[ 1.0000e+00,  4.9975e-19],
        [-1.2334e-15,  1.0000e+00]], device='cuda:0', dtype=torch.float64)
    \end{Verbatim}

    \(X \times K = Y \to K = X^{+} \times Y\), где \(X^{+}\) - матрица,
псевдообратная к \(X\)

    \begin{tcolorbox}[breakable, size=fbox, boxrule=1pt, pad at break*=1mm,colback=cellbackground, colframe=cellborder]
\prompt{In}{incolor}{94}{\boxspacing}
\begin{Verbatim}[commandchars=\\\{\}]
\PY{n}{K\PYZus{}approx} \PY{o}{=} \PY{n}{torch}\PY{o}{.}\PY{n}{mm}\PY{p}{(}\PY{n}{X\PYZus{}psi}\PY{p}{,} \PY{n}{Y\PYZus{}tensor}\PY{p}{)}
\PY{n+nb}{print}\PY{p}{(}\PY{n}{K\PYZus{}approx}\PY{p}{)}
\end{Verbatim}
\end{tcolorbox}

    \begin{Verbatim}[commandchars=\\\{\}]
tensor([[8.9280e-04],
        [9.9093e-01]], device='cuda:0', dtype=torch.float64)
    \end{Verbatim}

    \begin{tcolorbox}[breakable, size=fbox, boxrule=1pt, pad at break*=1mm,colback=cellbackground, colframe=cellborder]
\prompt{In}{incolor}{95}{\boxspacing}
\begin{Verbatim}[commandchars=\\\{\}]
\PY{n}{K} \PY{o}{=} \PY{n}{K\PYZus{}approx}\PY{o}{.}\PY{n}{cpu}\PY{p}{(}\PY{p}{)}

\PY{n}{Td} \PY{o}{=} \PY{l+m+mf}{0.001}
\PY{n}{L} \PY{o}{=} \PY{n}{Td} \PY{o}{/} \PY{n}{K}\PY{p}{[}\PY{l+m+mi}{0}\PY{p}{]}
\PY{n}{R} \PY{o}{=} \PY{p}{(}\PY{n}{L} \PY{o}{\PYZhy{}} \PY{n}{K}\PY{p}{[}\PY{l+m+mi}{1}\PY{p}{]} \PY{o}{*} \PY{n}{L}\PY{p}{)} \PY{o}{/} \PY{n}{Td}

\PY{n+nb}{print}\PY{p}{(}\PY{l+s+s1}{\PYZsq{}}\PY{l+s+s1}{Вычисленное значение R = }\PY{l+s+s1}{\PYZsq{}}\PY{p}{,} \PY{n}{R}\PY{o}{.}\PY{n}{numpy}\PY{p}{(}\PY{p}{)}\PY{p}{[}\PY{l+m+mi}{0}\PY{p}{]}\PY{p}{,} \PY{l+s+s1}{\PYZsq{}}\PY{l+s+s1}{ Ом}\PY{l+s+s1}{\PYZsq{}}\PY{p}{)}
\PY{n+nb}{print}\PY{p}{(}\PY{l+s+s1}{\PYZsq{}}\PY{l+s+s1}{Вычисленное значение L = }\PY{l+s+s1}{\PYZsq{}}\PY{p}{,} \PY{n}{L}\PY{o}{.}\PY{n}{numpy}\PY{p}{(}\PY{p}{)}\PY{p}{[}\PY{l+m+mi}{0}\PY{p}{]}\PY{p}{,} \PY{l+s+s1}{\PYZsq{}}\PY{l+s+s1}{ Гн}\PY{l+s+s1}{\PYZsq{}}\PY{p}{)}
\end{Verbatim}
\end{tcolorbox}

    \begin{Verbatim}[commandchars=\\\{\}]
Вычисленное значение R =  10.158316105496112  Ом
Вычисленное значение L =  1.1200701242066806  Гн
    \end{Verbatim}

    Также можно посчитать те же данные согласно формулам из исходной
лабораторной работы, которые я считаю не самыми корректными

    \begin{tcolorbox}[breakable, size=fbox, boxrule=1pt, pad at break*=1mm,colback=cellbackground, colframe=cellborder]
\prompt{In}{incolor}{96}{\boxspacing}
\begin{Verbatim}[commandchars=\\\{\}]
\PY{n}{R} \PY{o}{=} \PY{l+m+mi}{1} \PY{o}{/} \PY{n}{K}\PY{p}{[}\PY{l+m+mi}{0}\PY{p}{]} \PY{o}{*} \PY{p}{(}\PY{l+m+mi}{1} \PY{o}{\PYZhy{}} \PY{n}{K}\PY{p}{[}\PY{l+m+mi}{1}\PY{p}{]}\PY{p}{)}
\PY{n}{T} \PY{o}{=} \PY{o}{\PYZhy{}}\PY{n}{Td} \PY{o}{/} \PY{n}{np}\PY{o}{.}\PY{n}{log}\PY{p}{(}\PY{n}{K}\PY{p}{[}\PY{l+m+mi}{1}\PY{p}{]}\PY{p}{)}
\PY{n}{L} \PY{o}{=} \PY{n}{T} \PY{o}{*} \PY{n}{R}

\PY{n+nb}{print}\PY{p}{(}\PY{l+s+s1}{\PYZsq{}}\PY{l+s+s1}{R = }\PY{l+s+s1}{\PYZsq{}}\PY{p}{,} \PY{n}{R}\PY{o}{.}\PY{n}{numpy}\PY{p}{(}\PY{p}{)}\PY{p}{[}\PY{l+m+mi}{0}\PY{p}{]}\PY{p}{,} \PY{l+s+s1}{\PYZsq{}}\PY{l+s+s1}{ Ohm}\PY{l+s+s1}{\PYZsq{}}\PY{p}{)}
\PY{n+nb}{print}\PY{p}{(}\PY{l+s+s1}{\PYZsq{}}\PY{l+s+s1}{L = }\PY{l+s+s1}{\PYZsq{}}\PY{p}{,} \PY{n}{L}\PY{o}{.}\PY{n}{numpy}\PY{p}{(}\PY{p}{)}\PY{p}{[}\PY{l+m+mi}{0}\PY{p}{]}\PY{p}{,} \PY{l+s+s1}{\PYZsq{}}\PY{l+s+s1}{ Hn}\PY{l+s+s1}{\PYZsq{}}\PY{p}{)}
\end{Verbatim}
\end{tcolorbox}

    \begin{Verbatim}[commandchars=\\\{\}]
R =  10.158316105496143  Ohm
L =  1.1149832536880728  Hn
    \end{Verbatim}

    \hypertarget{ux440ux430ux441ux441ux447ux438ux442ux430ux442ux44c-ux441ux440ux435ux434ux43dux438ux435-ux437ux43dux430ux447ux435ux43dux438ux44f-ux438-ux441ux442ux430ux43dux434ux430ux440ux442ux43dux43eux435-ux43eux442ux43aux43bux43eux43dux435ux43dux438ux435.}{%
\subsubsection{5 Рассчитать средние значения и стандартное
отклонение.}\label{ux440ux430ux441ux441ux447ux438ux442ux430ux442ux44c-ux441ux440ux435ux434ux43dux438ux435-ux437ux43dux430ux447ux435ux43dux438ux44f-ux438-ux441ux442ux430ux43dux434ux430ux440ux442ux43dux43eux435-ux43eux442ux43aux43bux43eux43dux435ux43dux438ux435.}}

    Для нахождения ошибки между реальным значением \(Y\) и его предсказанием
моделью \(X \times K\), можно просто посчитать их разность
\(e = Y - X \times K\)

Тогда Сумма квадратов ошибки будет:

\[
S(K) = \sum e_{i}^{2} = e^{T} \times e = (Y - X \times K)^{T} \times (Y - X \times K)
\]

А среднеквадратичное отклонение:

\[
\sigma_{Y} = \sqrt {\dfrac {S(K)} {n} }
\]

    \begin{tcolorbox}[breakable, size=fbox, boxrule=1pt, pad at break*=1mm,colback=cellbackground, colframe=cellborder]
\prompt{In}{incolor}{97}{\boxspacing}
\begin{Verbatim}[commandchars=\\\{\}]
\PY{n}{e2\PYZus{}Y} \PY{o}{=} \PY{n}{torch}\PY{o}{.}\PY{n}{mm}\PY{p}{(}\PY{n}{Y\PYZus{}tensor}\PY{o}{.}\PY{n}{T} \PY{o}{\PYZhy{}} \PY{n}{torch}\PY{o}{.}\PY{n}{mm}\PY{p}{(}\PY{n}{X\PYZus{}tensor}\PY{p}{,} \PY{n}{K\PYZus{}approx}\PY{p}{)}\PY{o}{.}\PY{n}{T}\PY{p}{,} \PY{n}{Y\PYZus{}tensor} \PY{o}{\PYZhy{}} \PY{n}{torch}\PY{o}{.}\PY{n}{mm}\PY{p}{(}\PY{n}{X\PYZus{}tensor}\PY{p}{,} \PY{n}{K\PYZus{}approx}\PY{p}{)}\PY{p}{)}
\PY{n}{sigma2\PYZus{}Y} \PY{o}{=} \PY{n}{torch}\PY{o}{.}\PY{n}{divide}\PY{p}{(}\PY{n}{e2\PYZus{}Y}\PY{p}{,} \PY{n}{Y\PYZus{}tensor}\PY{o}{.}\PY{n}{shape}\PY{p}{[}\PY{l+m+mi}{0}\PY{p}{]}\PY{p}{)}

\PY{n}{sigma\PYZus{}Y} \PY{o}{=} \PY{n}{torch}\PY{o}{.}\PY{n}{sqrt}\PY{p}{(}\PY{n}{sigma2\PYZus{}Y}\PY{p}{)}
\PY{n}{sigma\PYZus{}Y} \PY{o}{=} \PY{n}{sigma\PYZus{}Y}\PY{o}{.}\PY{n}{cpu}\PY{p}{(}\PY{p}{)}\PY{o}{.}\PY{n}{numpy}\PY{p}{(}\PY{p}{)}\PY{p}{[}\PY{l+m+mi}{0}\PY{p}{]}\PY{p}{[}\PY{l+m+mi}{0}\PY{p}{]}

\PY{n+nb}{print}\PY{p}{(}\PY{n}{sigma\PYZus{}Y}\PY{p}{)}
\end{Verbatim}
\end{tcolorbox}

    \begin{Verbatim}[commandchars=\\\{\}]
0.00042367658795242503
    \end{Verbatim}

    \begin{tcolorbox}[breakable, size=fbox, boxrule=1pt, pad at break*=1mm,colback=cellbackground, colframe=cellborder]
\prompt{In}{incolor}{98}{\boxspacing}
\begin{Verbatim}[commandchars=\\\{\}]
\PY{n}{Y\PYZus{}predict} \PY{o}{=} \PY{n}{torch}\PY{o}{.}\PY{n}{mm}\PY{p}{(}\PY{n}{X\PYZus{}tensor}\PY{p}{,} \PY{n}{K\PYZus{}approx}\PY{p}{)}
\PY{n}{Y\PYZus{}predict} \PY{o}{=} \PY{n}{Y\PYZus{}predict}\PY{o}{.}\PY{n}{cpu}\PY{p}{(}\PY{p}{)}\PY{o}{.}\PY{n}{numpy}\PY{p}{(}\PY{p}{)}

\PY{c+c1}{\PYZsh{} print(Y\PYZus{}predict.T[0].shape)}
\PY{c+c1}{\PYZsh{} print(dataset\PYZus{}dict[\PYZdq{}current\PYZdq{}][1:].shape)}

\PY{n}{plt}\PY{o}{.}\PY{n}{plot}\PY{p}{(}\PY{n}{dataset\PYZus{}dict}\PY{p}{[}\PY{l+s+s2}{\PYZdq{}}\PY{l+s+s2}{time}\PY{l+s+s2}{\PYZdq{}}\PY{p}{]}\PY{p}{[}\PY{l+m+mi}{1}\PY{p}{:}\PY{p}{]}\PY{p}{,} \PY{n}{dataset\PYZus{}dict}\PY{p}{[}\PY{l+s+s2}{\PYZdq{}}\PY{l+s+s2}{current}\PY{l+s+s2}{\PYZdq{}}\PY{p}{]}\PY{p}{[}\PY{l+m+mi}{1}\PY{p}{:}\PY{p}{]}\PY{p}{,} \PY{l+s+s1}{\PYZsq{}}\PY{l+s+s1}{b}\PY{l+s+s1}{\PYZsq{}}\PY{p}{)}
\PY{n}{plt}\PY{o}{.}\PY{n}{plot}\PY{p}{(}\PY{n}{dataset\PYZus{}dict}\PY{p}{[}\PY{l+s+s2}{\PYZdq{}}\PY{l+s+s2}{time}\PY{l+s+s2}{\PYZdq{}}\PY{p}{]}\PY{p}{[}\PY{l+m+mi}{1}\PY{p}{:}\PY{p}{]}\PY{p}{,} \PY{n}{Y\PYZus{}predict}\PY{o}{.}\PY{n}{T}\PY{p}{[}\PY{l+m+mi}{0}\PY{p}{]}\PY{p}{,} \PY{l+s+s1}{\PYZsq{}}\PY{l+s+s1}{r\PYZhy{}\PYZhy{}}\PY{l+s+s1}{\PYZsq{}}\PY{p}{)}
\PY{n}{plt}\PY{o}{.}\PY{n}{xlim}\PY{p}{(}\PY{n}{time\PYZus{}interval}\PY{p}{)}
\PY{n}{plt}\PY{o}{.}\PY{n}{xlabel}\PY{p}{(}\PY{l+s+s1}{\PYZsq{}}\PY{l+s+s1}{Время, с}\PY{l+s+s1}{\PYZsq{}}\PY{p}{)}
\PY{n}{plt}\PY{o}{.}\PY{n}{ylabel}\PY{p}{(}\PY{l+s+s1}{\PYZsq{}}\PY{l+s+s1}{Сила Тока, А}\PY{l+s+s1}{\PYZsq{}}\PY{p}{)}
\PY{n}{plt}\PY{o}{.}\PY{n}{legend}\PY{p}{(}\PY{p}{[}\PY{l+s+s2}{\PYZdq{}}\PY{l+s+s2}{Реальное значение тока}\PY{l+s+s2}{\PYZdq{}}\PY{p}{,} \PY{l+s+s2}{\PYZdq{}}\PY{l+s+s2}{Предсказание модели}\PY{l+s+s2}{\PYZdq{}}\PY{p}{]}\PY{p}{)}
\PY{n}{plt}\PY{o}{.}\PY{n}{grid}\PY{p}{(}\PY{p}{)}
\end{Verbatim}
\end{tcolorbox}

    \begin{center}
    \adjustimage{max size={0.9\linewidth}{0.9\paperheight}}{lab1_files/lab1_29_0.png}
    \end{center}
    { \hspace*{\fill} \\}
    
    Как видно из графика, наша дискретная модель неплохо предсказывает
следующее значение силы тока по напряжению и силе тока на предыдущей
шаг, но на вопрос, насколько хорошо, способен ответить график ошибки
между реальным значением Y и его предсказанием:

    \begin{tcolorbox}[breakable, size=fbox, boxrule=1pt, pad at break*=1mm,colback=cellbackground, colframe=cellborder]
\prompt{In}{incolor}{99}{\boxspacing}
\begin{Verbatim}[commandchars=\\\{\}]
\PY{c+c1}{\PYZsh{} print(Y\PYZus{}predict.T[0].shape)}
\PY{c+c1}{\PYZsh{} print(dataset\PYZus{}dict[\PYZdq{}current\PYZdq{}][1:].shape)}

\PY{n}{max\PYZus{}offset} \PY{o}{=} \PY{n}{np}\PY{o}{.}\PY{n}{max}\PY{p}{(}\PY{n}{np}\PY{o}{.}\PY{n}{abs}\PY{p}{(}\PY{n}{dataset\PYZus{}dict}\PY{p}{[}\PY{l+s+s2}{\PYZdq{}}\PY{l+s+s2}{current}\PY{l+s+s2}{\PYZdq{}}\PY{p}{]}\PY{p}{[}\PY{l+m+mi}{1}\PY{p}{:}\PY{p}{]} \PY{o}{\PYZhy{}} \PY{n}{Y\PYZus{}predict}\PY{o}{.}\PY{n}{T}\PY{p}{[}\PY{l+m+mi}{0}\PY{p}{]}\PY{p}{)}\PY{p}{)}

\PY{n}{plt}\PY{o}{.}\PY{n}{plot}\PY{p}{(}\PY{n}{dataset\PYZus{}dict}\PY{p}{[}\PY{l+s+s2}{\PYZdq{}}\PY{l+s+s2}{time}\PY{l+s+s2}{\PYZdq{}}\PY{p}{]}\PY{p}{[}\PY{l+m+mi}{1}\PY{p}{:}\PY{p}{]}\PY{p}{,} \PY{n}{dataset\PYZus{}dict}\PY{p}{[}\PY{l+s+s2}{\PYZdq{}}\PY{l+s+s2}{current}\PY{l+s+s2}{\PYZdq{}}\PY{p}{]}\PY{p}{[}\PY{l+m+mi}{1}\PY{p}{:}\PY{p}{]} \PY{o}{\PYZhy{}} \PY{n}{Y\PYZus{}predict}\PY{o}{.}\PY{n}{T}\PY{p}{[}\PY{l+m+mi}{0}\PY{p}{]}\PY{p}{,} \PY{l+s+s1}{\PYZsq{}}\PY{l+s+s1}{g}\PY{l+s+s1}{\PYZsq{}}\PY{p}{)}
\PY{n}{plt}\PY{o}{.}\PY{n}{hlines}\PY{p}{(}\PY{p}{[}\PY{o}{\PYZhy{}}\PY{n}{sigma\PYZus{}Y}\PY{p}{,} \PY{n}{sigma\PYZus{}Y}\PY{p}{]}\PY{p}{,} \PY{n}{dataset\PYZus{}dict}\PY{p}{[}\PY{l+s+s2}{\PYZdq{}}\PY{l+s+s2}{time}\PY{l+s+s2}{\PYZdq{}}\PY{p}{]}\PY{p}{[}\PY{l+m+mi}{0}\PY{p}{]}\PY{p}{,} \PY{n}{dataset\PYZus{}dict}\PY{p}{[}\PY{l+s+s2}{\PYZdq{}}\PY{l+s+s2}{time}\PY{l+s+s2}{\PYZdq{}}\PY{p}{]}\PY{p}{[}\PY{o}{\PYZhy{}}\PY{l+m+mi}{1}\PY{p}{]}\PY{p}{,} \PY{l+s+s1}{\PYZsq{}}\PY{l+s+s1}{r}\PY{l+s+s1}{\PYZsq{}}\PY{p}{)}
\PY{n}{plt}\PY{o}{.}\PY{n}{hlines}\PY{p}{(}\PY{p}{[}\PY{o}{\PYZhy{}}\PY{n}{max\PYZus{}offset}\PY{p}{,} \PY{n}{max\PYZus{}offset}\PY{p}{]}\PY{p}{,} \PY{n}{dataset\PYZus{}dict}\PY{p}{[}\PY{l+s+s2}{\PYZdq{}}\PY{l+s+s2}{time}\PY{l+s+s2}{\PYZdq{}}\PY{p}{]}\PY{p}{[}\PY{l+m+mi}{0}\PY{p}{]}\PY{p}{,} \PY{n}{dataset\PYZus{}dict}\PY{p}{[}\PY{l+s+s2}{\PYZdq{}}\PY{l+s+s2}{time}\PY{l+s+s2}{\PYZdq{}}\PY{p}{]}\PY{p}{[}\PY{o}{\PYZhy{}}\PY{l+m+mi}{1}\PY{p}{]}\PY{p}{,} \PY{l+s+s1}{\PYZsq{}}\PY{l+s+s1}{b}\PY{l+s+s1}{\PYZsq{}}\PY{p}{)}
\PY{n}{plt}\PY{o}{.}\PY{n}{xlim}\PY{p}{(}\PY{n}{time\PYZus{}interval}\PY{p}{)}
\PY{n}{plt}\PY{o}{.}\PY{n}{xlabel}\PY{p}{(}\PY{l+s+s1}{\PYZsq{}}\PY{l+s+s1}{Время, с}\PY{l+s+s1}{\PYZsq{}}\PY{p}{)}
\PY{n}{plt}\PY{o}{.}\PY{n}{ylabel}\PY{p}{(}\PY{l+s+s1}{\PYZsq{}}\PY{l+s+s1}{Сила Тока, А}\PY{l+s+s1}{\PYZsq{}}\PY{p}{)}
\PY{n}{plt}\PY{o}{.}\PY{n}{legend}\PY{p}{(}\PY{p}{[}\PY{l+s+s2}{\PYZdq{}}\PY{l+s+s2}{Ошибка предсказания тока моделью}\PY{l+s+s2}{\PYZdq{}}\PY{p}{,} \PY{l+s+sa}{f}\PY{l+s+s2}{\PYZdq{}}\PY{l+s+s2}{Среднеквадратичное отклонение = }\PY{l+s+si}{\PYZob{}}\PY{n}{sigma\PYZus{}Y}\PY{l+s+si}{:}\PY{l+s+s2}{.6f}\PY{l+s+si}{\PYZcb{}}\PY{l+s+s2}{\PYZdq{}}\PY{p}{,} \PY{l+s+sa}{f}\PY{l+s+s2}{\PYZdq{}}\PY{l+s+s2}{Максимальное отклонение = }\PY{l+s+si}{\PYZob{}}\PY{n}{max\PYZus{}offset}\PY{l+s+si}{:}\PY{l+s+s2}{.6f}\PY{l+s+si}{\PYZcb{}}\PY{l+s+s2}{\PYZdq{}}\PY{p}{]}\PY{p}{)}
\PY{n}{plt}\PY{o}{.}\PY{n}{grid}\PY{p}{(}\PY{p}{)}
\end{Verbatim}
\end{tcolorbox}

    \begin{center}
    \adjustimage{max size={0.9\linewidth}{0.9\paperheight}}{lab1_files/lab1_31_0.png}
    \end{center}
    { \hspace*{\fill} \\}
    
    Как видно из графика ошибки предсказания, она находится по большей части
в границах среднеквадратичного отклонения, что говорит о корректности
вычисления \(\sigma_{Y}\), а максимальное отклонение находится не далеко
от среднеквадратичного, что говорит об отсутствии серьёзных выбросов в
исходных данных.


    % Add a bibliography block to the postdoc
    
    
    
\end{document}
